%\documentstyle[11pt,a4]{article}
%\documentclass[a4paper]{article}
\documentclass[a4paper, 10pt]{article}
% Seems like it does not support 9pt and less. Anyways I should stick to 10pt.
%\documentclass[a4paper, 9pt]{article}
\topmargin-2.0cm

\usepackage{fancyhdr}
\usepackage{pagecounting}
\usepackage[usenames,dvipsnames]{xcolor} 
\usepackage{comment}
\definecolor{gray}{rgb}{0.45,0.45,0.45}
\definecolor{red}{rgb}{.63, .235, .235}
\definecolor{myblue}{rgb}{0,0.75,1.0}
\usepackage[colorlinks,linkcolor=red,citecolor=red,urlcolor=red]{hyperref}
% Color Information from - http://www-h.eng.cam.ac.uk/help/tpl/textprocessing/latex_advanced/node13.html

% NEW COMMAND
% marginsize{left}{right}{top}{bottom}:
%\marginsize{3cm}{2cm}{1cm}{1cm}
%\marginsize{0.85in}{0.85in}{0.625in}{0.625in}

\advance\oddsidemargin-.96in
%\advance\evensidemargin-1.5cm
\textheight9.2in
\textwidth6.75in
\newcommand\bb[1]{\mbox{\em #1}}
\def\baselinestretch{1.05}
%\pagestyle{empty}

\newcommand{\hsp}{\hspace*{\parindent}}
%\definecolor{gray}{rgb}{1.0,1.0,1.0}

\newcommand{\ronghui}[1]{\textbf{\textcolor{myblue}{[ #1]}}}
\newcommand*{\email}[1]{\normalsize\href{mailto:#1}{#1}}

\begin{document}
\thispagestyle{fancy}
%\pagenumbering{gobble}
%\fancyhead[location]{text} 
% Leave Left and Right Header empty.
\lhead{}
\rhead{}
%\rhead{\thepage}
\renewcommand{\headrulewidth}{0pt} 
\renewcommand{\footrulewidth}{0pt} 
\fancyfoot[C]{\footnotesize \href{http://www.guronghui.com}{\color{gray}{http://www.guronghui.com}}} 

%\pagestyle{myheadings}
%\markboth{Sundar Iyer}{Sundar Iyer}

\pagestyle{fancy}
\lhead{\textcolor{gray}{\it Ronghui Gu}}
\rhead{\textcolor{gray}{\thepage/\totalpages{}}}
%\rhead{\thepage}
%\renewcommand{\headrulewidth}{0pt} 
%\renewcommand{\footrulewidth}{0pt} 
%\fancyfoot[C]{\footnotesize http://www.stanford.edu/$\sim$sundaes/application} 
%\ref{TotPages}

% This kind of makes 10pt to 9 pt.
\begin{small}

%\vspace*{0.1cm}
\begin{center}
{\LARGE \bf RESEARCH STATEMENT}\\
\vspace*{0.2cm}
{\normalsize Ronghui Gu (\email{ronghui.gu@yale.edu})}
\end{center}
%\vspace*{0.2cm}

%\begin{document}
%\centerline {\Large \bf Research Statement for Sundar Iyer}
%\vspace{0.5cm}

% Write about research interests...
%\footnotemark
%\footnotetext{Check This}

%[[What is my research goal?]]
My research goal is to make the software systems truly reliable and secure through \emph{formal verification}.
%[[Why?]]
As the backbone of 
modern software systems,
operating system (OS) kernels,
on the one hand,
 can have the greatest impact on the reliability and security of today's computing host.
On the other hand, OS kernels are complicated, highly-parallel, and prone to bugs. For the past several years,  my research has focused on developing scalable tools to build verified sequential and concurrent OS kernels that are formally proved to be error-free and secure~\cite{popl15-gu, osdi16-gu, pldi16-security, pldi16-device}.
Broadly speaking, my research
falls into the subfield of programming languages that deals with the 
%fundamental
principles and practice of the formal verification
of real software systems.


% Say that research work has been both theoritical and practical.

\subsubsection*{\large My Current Work on CertiKOS}

%[[Background, challenges: 3 points: cost, extensibility (new features + new properties), concurrency) ]]
While such mechanical ``formal verification" has a long history dating back to the 1960s, 
complete formal proofs of sequential OS kernels only became feasible recently, demonstrated by
seL4 in 2009. The result of seL4 was so encouraging that everyone
believed it was only a mile away to verify a real and concurrent OS kernel under reasonable efforts. However, seven years have passed,
and the situation is not likely to get better. 
The initial version of seL4 took 12 person-years to develop.
Even until now, the cost of such verification is still quite prohibitive,
 and it is unclear how to quickly adapt such a verified system to support new features and enforce richer properties. 
Before our work, none of the existing verified systems have addressed the important
issues of concurrency.
This last mile once seemed insurmountable.
%~\cite{apsys}.
%These problems severely limit
%the applicability and power of today's formally verified
%software systems.

%[[An overview of my approach, and a summary of my results]]
 
We believe that these verification problems
are caused by an overlook of the \emph{layered structure} in the proofs.
Although the system implementations consist of multiple
well-designed layers, this layered structure
is gone in the proof when using the verification techniques like  program logics.
Reasoning about the execution of kernel modules across different layers and among multiple threads is restricted within 
the same abstraction level. It makes the verification
of real software systems  difficult to untangle
and extend.

My thesis research is among the first to address these challenges
by exploring and realizing a novel class of specifications, named \emph{deep specifications}, through layered approach.
In theory, these layered deep specifications are rich
enough to fully characterize the functionality of the system implementation
and
uncover the fundamental
insights of the  layered design pattern. In practice, they are ``live" enough to  connect   with the actual  implementation directly and provide a modular approach to building
 software system stacks that are entirely trustworthy.
% even in the concurrent setting.
The advances in both dimensions have resulted in a comprehensive
verification framework, named CertiKOS, and
a series of fully verified sequential and concurrent OS kernels that were praised as ``a real breakthrough"~\cite{news, dnews}.
This CertiKOS work composes the main body of my thesis research, which is nominated for the ACM Doctoral Award,
and even inspires a decent research study of the science of deep specifications~\cite{deepspec}.


\subsubsection*{\small Deep Specifications and Certified Abstraction Layers}
\begin{comment}
Modern OS kernels are designed and constructed using a
stack of \emph{abstraction layers},
each of which defines an interface that hides the underlying implementation
details. Client programs can be understood solely based on the interface,
independent of the layer implementation. Despite their huge contributions
to the computer industry,
abstraction layers have mostly been treated as a system
concept; before our work, they have almost never been formally specified or verified.
\end{comment}
One innovation of my thesis research is that  OS kernels are treated as \emph{run-time compilers}~\cite{popl15-gu}.
From this novel view,
the OS kernel as a whole compiles the
user programs that are understood using system call specifications  to the programs that interact 
with the kernel implementation directly.
A layer in the kernel can be viewed as a \emph{compilation phase}
and the kernel module between two layers  is then a \emph{transformation}.
In this way,  OS kernels can be verified by showing that
every such transformation
preserves the behavior of arbitrary \emph{context} programs.
Since the preservation property holds for any context,
these layer interfaces are named
as deep specifications. 


As compilers enable program development
in higher-level languages that are machine-independent,
this layered approach allow program verification 
at some more abstract interfaces that are implementation-independent. Each kernel module is verified
at its proper abstraction level by showing the contextual
simulation relation between the implementation at that level
and the specification at a higher level.
Once this proof is done, the specification forms a new layer interface, which we call a \emph{certified abstraction layer}.
Any client program or any functional property of this module
can be reasoned about using the more abstract  layer specification alone, without going through the actual implementation.

To apply layer-based verification for real software systems,
we develop the CertiKOS framework in the Coq proof assistant
%~\cite{coq}
to specify, program, verify, and compose certified abstraction
layers. With CertiKOS, the verification task of OS kernels
can be mechanically decomposed into many simple, small, and
independent tasks, which are automatable and scalable.
To demonstrate the power of our framework,
 we have successfully developed multiple certified
sequential OS kernels in Coq. 
The most realistic one is called mCertiKOS,
which consists of 37 certified abstraction layers, 
took less than one person-year to develop,
and can boot a version of Linux as a guest.
An extended version of mCertiKOS was deployed on a military land vehicle in the context of a large DARPA-funded research project.

\subsubsection*{\small Verifying Concurrent OS Kernels with CertiKOS}
Moving from the sequential kernel verification
to the concurrent one is not straightforward at all.
A concurrent kernel allows interleaved
execution of kernel/user modules across different
layers of abstraction. 
The complete formal verification of a non-trivial concurrent
OS kernel is widely considered a grand challenge. 
Several researchers even believe that the combination of concurrency and the kernels' functional complexity makes the formal verification intractable,
and even if it is possible, the cost would far exceed that of
verifying a single-core sequential kernel.

We believe that the key to untangling this ``intractable" complexity roots in the
strong contextual property of deep specifications~\cite{osdi16-gu}.
A deep specification, in the concurrent setting, has to ensure   the behavior preservation not only under any client context but also under
any \emph{concurrent context} with any interleaving.
Each execution of the kernel corresponds to a  concurrent context, which is represented as a list of \emph{events} that encapsulates the behavior of the rest of CPUs (or threads), as well as the interference
among them under this execution.

The \emph{certified concurrent layer}  is then parameterized over this concurrent context.
Given a particular concurrent context,
the interleaving is determined,
and a concurrent layer is reduced to a sequential one, which allows us to apply standard techniques for
verifying sequential programs to build new concurrent layers.
A newly introduced concurrent layer becomes ``deep" only after we show that its simulation proof holds for all valid concurrent contexts.
To ease this step, we creatively lift the concurrent machine model (which allows arbitrary interleaving at any point) to an abstract local machine model,
where all the impacts of the concurrent context are restricted at
some certain points, i.e.,
the invocations of synchronization primitives. This machine-lifting idea
enables the \emph{local reasoning} of multicore (and multithreaded) programs,
without worrying about the concurrent interleaving except for a few specific places. Furthermore, this machine lifting also guarantees that the local reasoning of all the
CPUs (and threads) can be composed together
and then 
propagated down to the execution of actual 
implementations over the  multicore hardware.

 
With this concurrent framework,
we have also successfully developed and verified a practical concurrent OS
kernel in Coq. Our certified kernel is written in 6,500 lines of
C and x86 assembly and runs on stock x86 multicore
machines. To our knowledge, this is the first proof of
(contextual) functional correctness and progress
property of a complete, general-purpose
concurrent OS kernel with fine-grained locking.
As quoted from the Yale News, this is ``a milestone that the scientists say could lead to a new generation of reliable and secure systems software"~\cite{news}.

\subsubsection*{\small Adapt CertiKOS to Verify Interruptable OS Kernels and Device Drivers}
In a monolithic kernel,
device drivers are not only the majority of the code base
but also the primary source of the system crashes.
Although formal verification of device drivers is highly desirable, it is widely considered as challenging, due to the abundance of device features
and the non-local effects of interrupts.

To address this issue, we introduce
a general device model that can be
instantiated with various hardware devices and a realistic
interrupt model that scales the reasoning
about interruptible code~\cite{pldi16-device}. 
The device drivers are modeled
as if each of them were running on a  separate \emph{logical CPU}. This novel idea allows us to
incrementally refine a raw
device into more and more abstract devices
by building certified layers of the relevant driver on its logical CPU.
Meanwhile, this idea systematically enforces
the isolation among different devices and the
rest of the kernel. This strong isolation property 
leads to an abstract interrupt model such that
most of the kernel execution is interrupt-unaware.

We have successfully extended the CertiKOS framework with these new models
and turned mCertiKOS
%from an verified non-interruptible kernel 
into a verified interruptible kernel with device
drivers, e.g., serial, IOAPIC, etc. 
The entire extension is realized in Coq and took roughly seven person-months.
To the best of our knowledge, this is the first verified
interruptible operating system with device drivers.

\subsubsection*{\small End-to-End Security Verification in CertiKOS}
Protecting the confidentiality of information manipulated by
a software system is one of the critical challenges
facing today's cybersecurity community. A promising step
toward conquering this challenge is to  verify  the
end-to-end information-flow
security of the whole system formally. 

This step can be easily established
using our CertiKOS framework~\cite{pldi16-security}. An example is our work proving
that mCertiKOS is end-to-end secure. We verified the noninterference between user processes 
only at the top-most layer of mCertiKOS, but the nature of deep specifications propagates
this security guarantee down to the concrete implementation.
The security guarantee of mCertiKOS
can be seen as end-to-end in the following two ways: (1)
it applies across all the layers; and (2) it ensures that the entire
execution is secure from start to finish.

\subsubsection*{\large Future Research Agenda}
The goal of my research is to
integrate the efficient and scalable formal verification
into the build of real software systems
and dramatically improve the software reliability and security. 
I plan to pursue this goal through the following three research directions at different stages.


\subsubsection*{\small Programming Certified Software Systems with CertiKOS Directly}
Existing  projects on real system verification,
including CertiKOS, all require   (manually)  writing the actual implementation
 in a C-like language and a formal specification in a proof assistant language.
Lacking the support of directly writing certified  programs  makes it difficult to develop and maintain certified  software systems
at scale. I aim to fill in this enormous gap between low-level system programming and high-level specification reasoning
by
providing
a uniform way to program certified software directly in CertiKOS.

This short term goal is ambitious  but still promising.
Due to the fact the all certified layers in CertiKOS
are ``deep", writing the layer specification alone is enough
to express the full functionality. I plan to extend the CertiKOS
framework such that the layer specification can be smartly 
``compiled" into C programs that meet the specification
and the layers at different abstract level can be linked together in a mostly automated way. Thanks to the layered approach of CertiKOS,
when focusing on a single verification task between two layers,
 the gap between the implementation and its deep specification, as well as the gap between the abstraction of two layer interfaces,
are not that big. I believe that we could generate
certified code from our deep specifications
follow the line of the program synthesis work
and could link layers at different levels automatically
by taking advantage of the recent progress in
artificial intelligence[]. We have successfully generated
the page allocator module of mCertiKOS. But there is
still a tremendous research opportunity for complete functional
synthesis and creating programs that are intelligent enough
to do formal proofs between layers.

\subsubsection*{\small Building a Zero-Vulnerability System Stack}
Once the verification framework allows writing certified programs in a more direct and efficient way, I aim to utilize it to build a system stack that is entirely trustworthy. Based on the hardware that supports cryptography primitives, I want to
establish a reliable and secure stack that integrates the device drivers,
database system, file system, network stacks, operating system,
 and trustable applications.
This zero-vulnerability system stack will dramatically improve
the reliability and security of computing hosts in the critical areas. 
However, there are two significant challenges to building such a stack:
(1)  systems at different levels may have various
reliability requirements, some of which are still unclear how
to specify and verify (e.g., the network stacks); and (2)
it is a big unknown how to link
all the certified systems together in a single framework
and provide the guarantee of the entire stack.

Our device driver verification work~\cite{pldi16-device} shows that
it is promising to address these challenges based on the CertiKOS framework. For different systems, we could
provide ``customized" machine models
by exposing a particular set of hardware features
plus an abstract interface of the underlying systems.
It allows us to build certified layers for each system
in an isolated way and the  composition rules of CertiKOS
could be strengthened to link all these layers. My Yale colleagues
and I have already started to use CertiKOS to verify more systems in this stack,
such as the file system and the network stacks.
Meanwhile, I also work with  a research group from UPenn to
set up a specification-based testing framework
that enables the random test for user-level applications upon the topmost layer interface of mCertiKOS.
I believe there is a huge research opportunity that
lies in this zero-vulnerability system stack
and I am looking forward to exploring it.


\subsubsection*{\small Pushing Towards  an Industry-Scale Verification Framework}
In the future, I aim to 
extend our verification framework such that
the  zero-vulnerability system stack can scale to the industry level. We not only have to provide a practical and powerful
set of standard certified libraries but also need to redesign
the specification language such that
it could precisely and efficiently describe the functionality
and is acceptable by average engineers.
If successful,
this research will make
the real software systems truly reliable and secure, making  a profound impact on the software industry
 and the society in general.
 
\vspace{0.5cm}
%\begin{flushright}
%Sundar Iyer
%\end{flushright}

\end{small}
%\newpage

%\begin{thebibliography}{deSolaPITH}
% Change font size?
% \tiny, \footnotesize, \small,\normalsize, \large, \Large, \LARGE, and \huge 
%\begin{small}
\begin{footnotesize}
\begin{thebibliography}{}
\bibliographystyle{}



\bibitem[POPL'15]{popl15-gu}
R. Gu, J. Koenig, T. Ramananandro, Z. Shao, X. Wu, S. Weng, H. Zhang, and Y. Guo. ``Deep specifications and certified abstraction layers." In
\emph{42nd ACM Symposium on Principles of Programming
Languages (POPL'15)}.

\bibitem[OSDI'16]{osdi16-gu}
R. Gu, Z. Shao, H. Chen, X. Wu, J. Kim, V. Sj\"{o}berg, and D. Costanzo. ``CertiKOS: An Extensible Architecture for Building Certified Concurrent OS Kernels." In \emph{12th USENIX Symposium on Operating Systems Design and Implementation (OSDI'16)}. 

\bibitem[PLDI'16a]{pldi16-security}
D. Costanzo,  Z. Shao, and R. Gu. ``End-to-end verification of information-flow security for C and assembly programs." In 
\emph{37th ACM SIGPLAN Conference on Programming Language Design and Implementation (PLDI'16)}.

\bibitem[PLDI'16b]{pldi16-device}
H. Chen,  X. Wu, Z. Shao, J. Lockerman, and R. Gu. ``Toward compositional verification of interruptible OS kernels and device drivers." In \emph{37th ACM SIGPLAN Conference on Programming Language Design and Implementation (PLDI'16)}.

%\bibitem[7]{state}
%S. Iyer, N. McKeown,
%``Maintaining State in Router Line Cards". In preparation for {\it IEEE Communication Letters}.

% VIMP - Change to TR report.

\bibitem[YaleNews]{news}
``CertiKOS: A breakthrough toward hacker-resistant operating systems." \emph{Yale News}, 2016.
\url{http://news.yale.edu/2016/11/14/certikos-breakthrough-toward-hacker-resistant-operating-systems}.

\bibitem[YaleDailyNews]{dnews}
``Yale computer scientists unveil new OS." \emph{Yale Daily News}, 2016.
\url{http://yaledailynews.com/blog/2016/11/18/yale-computer-scientists-unveil-new-os/}.

%S. Iyer, R. R. Kompella, N. McKeown, ``Analysis of a Memory Architecture for Fast Packet
%Buffers". Preliminary version in {\it IEEE - High Performance Switching and Routing}, Dallas,
%May 2001, pp. 368-373. Final version submitted for publication to {\it IEEE/ACM Transactions on
%Networking}, and is available at 
%http://yuba.stanford.edu/$\sim$sundaes/Papers/buffersubmit.pdf

\bibitem[DeepSpec]{deepspec}
DeepSpec: The science of deep specifications. 
\url{http://deepspec.org/}.

\begin{comment}
\bibitem[7]{sel4}
G. Klein, K. Elphinstone, G. Heiser, J. Andronick,
D. Cock, P. Derrin, D. Elkaduwe, K. Engelhardt,
R. Kolanski, M. Norrish, T. Sewell, H. Tuch, and
S. Winwood. ``seL4: Formal verification of an OS
kernel." In 22nd ACM Symposium on Operating
Systems Principles (SOSP 09).

\bibitem[8]{fscq}
H. Chen, D. Ziegler, T. Chajed, A. Chlipala, M. F.
Kaashoek, and N. Zeldovich. ``Using Crash Hoare
logic for certifying the FSCQ file system." In
25th ACM Symposium on Operating System Principles
(SOSP 15).

\bibitem[9]{apsys}
S. Peters, A. Danis, K. Elphinstone, and G. Heiser.
``For a microkernel, a big lock is fine." In 
Asia Pacific Workshop on Systems (APSys 15).

\bibitem[10]{coq}
The Coq proof assistant. The Coq development team.  \url{http://coq.
inria.fr}.
\end{comment}

%\bibitem[CoffmanO]{CoffmanO}
%K. G. Coffman and A. M. Odlyzko, The size and growth rate of the Internet.
%{\em First Monday,} Oct. 1998, $\langle$http://firstmonday.org/$\rangle$.
%Also available at $\langle$http://www.research.att.com/$\sim$amo$\rangle$.
%\bibitem[Dunn]{Dunn}
%L. Dunn, The Internet2 project, {The Internet Protocol Journal,} vol. 2,
%no. 4 (Dec. 1999).  Available at
%$\langle$http://www.cisco.com/warp/public/759/ipj\_issues.html$\rangle$.


\end{thebibliography}
\end{footnotesize}

\end{document}

